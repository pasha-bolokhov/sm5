%%%%%%%%%%%%%%%%%%%%%%%%%%%%%%%%%%%%%%%%%%%%%%%%%%%%%%%%%%%%%%%%%%%%%
%%%%  %%%%    %%%    %%%     %% %%%% %%    %%%%     %%% %%%% %%%%%%%%
%%% %% %%% %%% %% %%% %% %%%%%%  %%% %%% %% %%%%% %%%%%% %% %%%%%%%%%
%% %%%% %% %%% %% %%% %% %%%%%% % %% %%% %%% %%%% %%%%%%%  %%%%%%%%%%
%%      %%    %%%    %%%    %%% %% % %%% %%% %%%% %%%%%%%  %%%%%%%%%%
%% %%%% %% %%%%%% %%%%%% %%%%%% %%%  %%% %% %%%%% %%%%%% %% %%%%%%%%%
%% %%%% %% %%%%%% %%%%%%     %% %%%% %%    %%%%     %%% %%%% %%%%%%%%
%%%%%%%%%%%%%%%%%%%%%%%%%%%%%%%%%%%%%%%%%%%%%%%%%%%%%%%%%%%%%%%%%%%%%
\appendix

%%%%%%%%%%%%%%%%%%%%%%%%%%%%%%%%%%%%%%%%%%%%%%%%%%%%%%%%%%%%%%%%%%%%%%
%%%%%%%%%%%%%%%%%%%%%%%%%%%%%%%%%%%%%%%%%%%%%%%%%%%%%%%%%%%%%%%%%%%%%%
%
%                 INTRODUCTION TO LORENTZ VIOLATION
%
%%%%%%%%%%%%%%%%%%%%%%%%%%%%%%%%%%%%%%%%%%%%%%%%%%%%%%%%%%%%%%%%%%%%%%
%%%%%%%%%%%%%%%%%%%%%%%%%%%%%%%%%%%%%%%%%%%%%%%%%%%%%%%%%%%%%%%%%%%%%%
\section{RG equations for Dimension Five operators 
%	RG equations for Dimension Five LV operators in the Standard Model
	above the Electroweak symmetry breaking scale}

	Above the Electroweak symmetry breaking scale, one has all
	quark species and all lepton species present, and for
	simplicity can take Higgs to ``emerge'' at this scale too.
	In the matrix form, one-loop RG equations for LV operators take the
	form
	(the notations for the Wilson coefficients are introduced in 
	section \ref{SM_3rank})
\begin{align}
% first line
\notag
	\mu \frac{\p}{\p\mu} 
		\left \lgroup 
			\begin{matrix}
				\eta_L \\
			   	\eta_Q \\
				\eta_e \\
				\eta_u \\
				\eta_d
			\end{matrix} 
		\right \rgroup & =
	\left[~
		( A ~+~ G )\;
		{\rm diag}
		\begin{Bmatrix}
			g'^2(\mu)\, Y_L^2  ~+~  g^2(\mu)\, T_W  \\
			g'^2(\mu)\, Y_Q^2  ~+~  g^2(\mu)\, T_W  ~+~  g_3^2(\mu)\, T_S\\
			g'^2(\mu)\, Y_e^2  \\
			g'^2(\mu)\, Y_u^2  ~+~    g_3^2(\mu)\, T_S \\
			g'^2(\mu)\, Y_d^2  ~+~    g_3^2(\mu)\, T_S \\
		\end{Bmatrix}
	\right.
		~+~~ \\
% second line
\notag
	&
	\left.
	\qquad\qquad\qquad+~~
		\frac{1}{2} G_H \;
		{\rm diag}
		\begin{Bmatrix}
		    \left\{\, \lambda_e \lambda_e^\dag  \,,\, \;\cdot\; \,\right\} \\
		    \left\{\, \lambda_d \lambda_d^\dag + 
					\lambda_u \lambda_u^\dag \,,\, \;\cdot\; \,\right\} \\
		    \left\{\, 2\,\lambda_e^\dag \lambda_e  \,,\, \;\cdot\; \,\right\} \\
		    \left\{\, 2\,\lambda_u^\dag \lambda_u  \,,\, \;\cdot\; \,\right\} \\
		    \left\{\, 2\,\lambda_d^\dag \lambda_d  \,,\, \;\cdot\; \,\right\} \\
		\end{Bmatrix}
	~\right] 
	\times
	\left\lgroup
	\begin{matrix}
		\eta_L \\
	   	\eta_Q \\
		\eta_e \\
		\eta_u \\
		\eta_d
	\end{matrix}
	\right\rgroup
	~~~+
	\\
% third line
\notag
	&
	+~~~
	K\, g'^2\, 
		\left \lgroup
		\begin{matrix}
			- Y_L^2 \\
			- Y_Q^2 \\
			+ Y_e^2 \\
			+ Y_u^2 \\
			+ Y_d^2
		\end{matrix}
		\right \rgroup
	\xi' (\mu) \cdot \mathbf{1}_{\rm flavor}
	~~~+~~~	
	K\, g^2(\mu)\,
		\left \lgroup
		\begin{matrix}
			- T_W \\
			- T_W \\
			   0  \\
			   0  \\
			   0 
		\end{matrix}
		\right \rgroup
	\xi(\mu) \cdot \mathbf{1}_{\rm flavor}
	\\
% fourth line
\notag
	&
	+~~~
	K\, g_3^2\, 
		\left \lgroup
		\begin{matrix}
			    0   \\
			- T_S   \\
			    0   \\
			  T_S   \\
			  T_S  
		\end{matrix}
		\right \rgroup
	\xi_3 (\mu) \cdot \mathbf{1}_{\rm flavor}
	~~~+~~~
	A_H\,
	\left \lgroup
		\begin{matrix}
			\lambda_e\, \eta_e\, \lambda_e^\dag \\
			\lambda_d\, \eta_d\, \lambda_d^\dag ~+~
				\lambda_u\, \eta_u\, \lambda_u^\dag \\
			2\, \lambda_e^\dag\, \eta_L\, \lambda_e \\
			2\, \lambda_u^\dag\, \eta_Q\, \lambda_u \\
			2\, \lambda_d^\dag\, \eta_Q\, \lambda_d \\
		\end{matrix}
	\right \rgroup
	\\
% fifth line
\notag
	&
	\qquad\qquad\qquad\qquad\qquad\qquad\qquad
	+~~~
	K_H\,
	\left \lgroup
		\begin{matrix}
			\lambda_e\, \lambda_e^\dag \\
			\lambda_d\, \lambda_d^\dag ~-~ 
				\lambda_u\, \lambda_u^\dag \\
			- 2\, \lambda_e^\dag \lambda_e \\
			+ 2\, \lambda_u^\dag \lambda_u \\
			- 2\, \lambda_d^\dag \lambda_d \\
		\end{matrix}
	\right \rgroup
	\cdot \kappa(\mu)~.
%
%
% third line
%
%	
%		\left\lgroup 
%			g'^2 ( \mu )\, {\rm diag} 
%				\left\lbrace
%					\begin{matrix}
%						Y_L^2 \\
%						Y_Q^2 \\
%						Y_e^2 \\
%						Y_u^2 \\
%						Y_d^2 
%					\end{matrix}
%				\right\rbrace
%			~+~
%			g^2 (\mu)\, {\rm diag}
%				\left\lbrace
%					\begin{matrix}
%						T_W   \\
%						T_W   \\
%						0     \\
%						0     \\
%						0     
%					\end{matrix}
%				\right\rbrace
%			~+~
%			g_3^2 (\mu)\, {\rm diag}
%				\left\lbrace
%					\begin{matrix}
%						0     \\
%						T_S   \\
%						0     \\
%						T_S   \\
%						T_S   
%					\end{matrix}
%				\right\rbrace
%		\right\rgroup
\end{align}
%
\begin{multline}
% first line
\notag
	\mu \frac{\p}{\p\mu} 
	\xi' ~~=~~
	\frac{1}{2}\, ( - L )\, (g')^2 
	\left\lgroup
		-2 Y_L^2,\, -2 N_S Y_Q^2,\, Y_e^2,\, N_S Y_u^2,\, N_S Y_d^2
	\right\rgroup
	\times
	\left\lgroup
	\begin{matrix}
		{\rm tr~}\eta_L \\
		{\rm tr~}\eta_Q \\
		{\rm tr~}\eta_e \\	
		{\rm tr~}\eta_u \\
		{\rm tr~}\eta_d
	\end{matrix}
	\right\rgroup
	~~+
	\\
% second line
\notag
%	&
	+~~~
	\alpha_1\cdot (g')^2 \cdot \xi'(\mu)
\end{multline}
%
\begin{equation}
% first line
\notag
	\mu \frac{\p}{\p\mu} 
	\xi 
	~=~
	\frac{1}{2}\, C_1\, L\, g^2 
	\left\lgroup 
	1\,,~ N_S
	\right\rgroup
	\left\lgroup
	\begin{matrix}
		{\rm tr~}\eta_L \\
		{\rm tr~}\eta_Q 
	\end{matrix}
	\right\rgroup
	~~+~~
	\left(
		\alpha_2\, g^2 
		~-~
		\frac{1}{2}
		Q_Y\, N_W\, g^2 
	\right)
	\cdot \xi (\mu)
\end{equation}
%
\begin{equation}
% first line
\notag
	\mu \frac{\p}{\p\mu} 
	\xi_3 
	~=~
	- \frac{1}{2}\, C_1\, L\, g_3^2 
		\left\lgroup
			-2\,,~1\,,~1
		\right\rgroup
	\left\lgroup
	\begin{matrix}
		{\rm tr~}\eta_Q \\
		{\rm tr~}\eta_u \\
		{\rm tr~}\eta_d
	\end{matrix}
	\right\rgroup
	~~+~~
	\left(
		\alpha_3\, g_3^2 
		~-~
		\frac{1}{2}
		Q_Y\, N_S\, g_3^2
	\right)
	\cdot \xi_3 (\mu)
\end{equation}
%
\begin{align}
% first line
\notag
	\mu \frac{\p}{\p\mu} \,
	\kappa 
	~=~&
	( A^\kappa ~+~ G^\kappa )
	\left[
		(g')^2\, Y_H^2 ~+~ g^2\, T_W
	\right] \cdot \kappa
	~~+ 
	\\
% second line
\notag
	&
	+~~
	\frac{1}{2}\, F_H\, 
	{\rm tr} 
	\left[
		\lambda_e\, \lambda_e^\dag
		~+~
		N_S\, \lambda_u\, \lambda_u^\dag
		~+~
		N_S\, \lambda_d\, \lambda_d^\dag
	\right]
	\cdot \kappa
	~~+
	\\
% third line
\notag
	&
	+~~
	\frac{1}{2}
	L_H\,
	{\rm tr} 
	\left[
		N_S\, \lambda_d^\dag\, \eta_Q\, \lambda_d 
		~+~
		\lambda_e^\dag\, \eta_L\, \lambda_e
		~-~
		N_S\, \lambda_u^\dag\, \eta_Q\, \lambda_u
		~-
	\right.
	\\
% fourth line
\notag
	&
	\qquad\qquad
	\left.
		-~
		N_S\, \lambda_d\, \eta_d\, \lambda_d^\dag
		~-~
		\lambda_e\, \eta_e\, \lambda_e^\dag
		~+~
		N_S\, \lambda_u\, \eta_u\, \lambda_u^\dag
	\right]
	~.
\end{align}

	Here the RG coefficients are
%
% designations of RG coefficients of dimension five operators
\begin{align}
\notag
	A & = \phantom{-}\frac{19~\;}{48\pi^2}  & A_H      & = - \frac{1}{96\pi^2} \\
\notag
	G & = \phantom{-}\frac{1}{8\pi^2}     &	G_H 	 & = \phantom{-}
							\frac{1}{16\pi^2}	\\
\notag
	L & = - \frac{1}{48\pi^2}  	      &	L_H 	 & = - \frac{1}{6\pi^2}  \\
\notag
	K & = - \frac{5~}{48\pi^2}  	      &	K_H	 & = - \frac{1}{64\pi^2} \\
\notag
	Q_{YM} & =
		- \frac{7}{6\pi^2} 	      &	F_H & = \phantom{-}\frac{1}{4\pi^2} \\
\notag
	A^\kappa & = \phantom{-}
	          \frac{2}{3\pi^2} 	      &   G^\kappa & = - \frac{1}{4\pi^2} 
	~.
\end{align}
	The running couplings are
%
% running couplings
\begin{align}
\notag
	\alpha_1 & =        
			    \frac{5/3\, N_g ~+~ 1/8}
                                        {6\pi^2}  \\
\notag
	\alpha_2 & =    -\, \frac{19 ~-~ 8 N_g}
		                  {48\pi^2}  \\
\notag
	\alpha_3 & =    -\, \frac{5 ~-~4/3\, N_g}
			         {8\pi^2}~,
\end{align}
	where $ N_g = 3 $ is the number of generations.
	We have used the following notations for the group theoretical
	factors:
\begin{align}
\notag
	& N_W = ~2 \quad\text{(dim fund SU(2))} &
		T_W & = \frac{N_W^2 ~-~ 1}
			       {2 N_W}       
	\\
\notag
	& N_S = ~3 \quad\text{(dim fund SU(3))} &
		T_S & = \frac{N_S^2 ~-~ 1}
			       {2 N_S}
	\\
\notag
	& {\rm tr} \left\lgroup T^a T^b \right\rgroup = 
			~C_1 \delta^{ab} &
		C_1 & \equiv \frac{1}{2}~.
\end{align}

	Below the EW symmetry breaking scale, one misses the gauge bosons
	and the Higgs.
	Effectively, one has only QED and QCD present.
	We take a naive ``stepwise'' account for treating mass threshold for quarks,
	whereby below its mass scale, the corresponding quark is considered
	to be absent (the same way Higgs and the gauge bosons are).
	
	At the EW scale, all SU(2) doublets break up into pairs of particles.
	So do the LV operators coupled to these doublets --- one now has
	separate operators for left neutrinos and for left electrons.
	The boundary condition is that at the EW scale the corresponding LV
	interactions ``unite'' into doublets:
\begin{align}
	&& \eta^L_{(\nu)} \Bigr|_{M_W} ~=~ \eta^L_{(e)} \Bigr|_{M_W} 
		~=~ \eta_L \Bigr|_{M_W}~,
	&& \eta^L_{(u)} \Bigr|_{M_W} ~=~ \eta^L_{(d)} \Bigr|_{M_W} 
		~=~ \eta_Q \Bigr|_{M_W}~.
\end{align}
	Righthanded particles of course continue to be independent.
	The Wilson coefficients are still matrices in the flavor space, but
	whenever a particle falls out of the spectrum, the corresponding
	matrices ``lose'' the subspace pertinent to that particle.
	However, now these matrices are in the mass basis rather than
	in the gauge basis {\bf (!!Modify the above equation)} ---
	this is what makes this flavor space ``reduction'' straightforward.

	In the gauge sector, as described in section \ref{SM_3rank}, 
	one has operators $ \xi_{\rm EM} $ and $ \xi_3 $, where 
	the former one is defined at the EW scale as
\[
	\xi_{\rm EM}\Bigr|_{M_W} ~=~ \xi'\Bigr|_{M_W} \cos^2 \theta_W ~+~ 
			\xi\Bigr|_{M_W} \sin^2 \theta_W~.
\]
	

	

\end{document}
