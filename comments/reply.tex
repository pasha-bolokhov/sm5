%\documentclass[12pt]{article}
\documentclass[12pt,preprintnumbers,nofootinbib]{revtex4}

\begin{document}

\begin{center}
\centering
\Large
Reply to the Comments
\end{center}

We would like to thank the referee for the thorough reading of the 
manuscript, and his/her tolerance towards our [obviously needed much 
improvement] wording in the text. 

We appreciate the main point that one cannot claim that all operators are
excluded at a very high level of precision, because there are very many of them, much 
more than the number of most precise observables. To that effect we added 
several sentences along the way, that somewhat soften our claims, opening 
section IV with the paragraph where we explain that the "limits" should be 
looked at more like the limits of sensitivity and some combinations of operators may indeed
escape the most prohibitive constraints. 

The replies to the specific concerns are given below. 

\begin{enumerate}
\item Page 2, ``The intriguing possibility ...'':
 This sentence needs to cite something,
 such as Kosteleck\'y and Samuel, PRD 39 683 (1989).

Reply. The citation indeed is needed and has been added.

\item Page 2, ``Dimension three Lorentz...'':
 The terms shown use incorrect notation.
 The left $\psi$ in each term should be $\bar{\psi}$,
 and the factors of $i$ shouldn't be there.

Reply. The terms have been corrected.


\item Page 2, ``This is, of course, ...'':
 There are two problems with this sentence:
 \begin{enumerate}
 \item A citation is needed for the experimental bound quoted.

Reply. Citation added. 

 \item The actual bounds that have been achieved directly involve nucleons, not
quarks.
 These probably induce bounds on quark parameters,
 but the exact nature of the quark bounds is an open question.

Reply. Further in the text we have mentioned that there is an
uncertainty in the induced LV parameters for quarks derived from the
nucleon constraints resulting from limited knowledge of the 
corresponding matrix elements. These uncertainties, we believe, are
acceptable for the main purpose of the paper.
 \end{enumerate}

\item Page 7, ``Similarly, $f_{1,2}^{\mu\nu}$ cannot admix...'':
 This is a very awkward sentence and should be rewritten.

Reply. The sentence has been changed.

\item Page 14, ``Such interactions are dangerous...'':
 The second half of this sentence is obviously false --
 the authors discuss the dangerous interactions for the rest of the page,
 including three more numbered equations.
 They should not claim to exclude anything until they actually do so.

Reply. This sentence has been corrected, first we identify the dangerous
operators, then they are really excluded.

\item Page 14, equation (17):
 This equation has several problems.
 \begin{enumerate}
 \item Current experiments place different limits on the $b^\mu$ parameter
 for protons, neutrons, and electrons.
 Equation (17) may apply to at most one of these.

Reply.
	We have added the citations where the limits were taken from.
	Limits for different species come from different sources.
	For the purpose of the argument, it is sufficient to use the
	common figure applicable to all species

 \item The bound certainly isn't taken from citation [18] (Gagnon and Moore),
 which only deals with dimension-four-and-higher operators.
 It seems to come from a low-energy experiment.
 {\itshape Which} experiment is comes from depends on which particle species
 it's intended to refer to.

Reply.
	Gagnon and Moore was referenced by mistake. Again, the citations
	show the proper references.
 \end{enumerate}

\item Page 14, ``Even in the very conservative assumption...'':
 It is not clear why the authors call potential efforts ``futile''.
 Is is because any effects associated with $\tilde{c}^\mu$ would be overwhelmed
 by other Lorentz-violating effects?
 Is is because the bound is far from rough estimates of the expected size of
$\tilde{c}^\mu$?
 The first reason is valid, but if true needs to be explained by the authors.
 The second reason is not valid because any estimated sizes are {\itshape very}
rough and unreliable.
 If either reason or something else pertains, it needs to be explained.

Reply. 
%We believe that the inferrable constraints make 
%LV induced by the given operators inaccessible by experiment, as the text
%now states. Direct detection of LV assumes {\it e.g.} measurement of interaction
%of nuclear spin with preferred directions, or with nuclear EM/strong fields, etc.
%The achievable level of measurement of such interactions is much lower than the
%constraints.
We agree that this is the case of awkward warding and we deleted "futile" 
sentence altogether. 


\item Page 16, Table 1:
 As with equation (17), the size of constraints depends on particle species.
 Moreover, many constraints deal directly with composite particles
 rather than quarks,
 and the full connection between the two is currently unknown.

Reply. The main purpose of the table is to give the ``order of magnitude''
constraints applicable to {\it groups} of operators, and the source where
they are coming from. We have mentioned that renormalization group mixing can 
be used to transfer constraints from one species to others. 
The comment regarding composite particles should refer to some specific group(s), 
since the constraints, say, for leptons coming from high-energy cosmic rays 
do not depend on unknown nucleon parameters.
In any case, the same way as with remark 3, we presume that the existing 
uncertainty is acceptable, what we now have mentioned in the text.

\item Page 17, ``Known limits [20] on interaction...'':
 While the cited reference is relevant,
 stronger bounds on nuclear Lorentz-violating interactions
 arise from several more-recent experiments.
 I suggest that the authors investigate those.

Reply. Proper reference has been added.

\item Page 17, ``We comment that ...'':
 This point is true, and is important enough that is should be mentioned
earlier.
 Among other places, something like it should appear in the caption to Table 1
 so that readers who only skim the current paper do not get the wrong
impression.  

Reply. To help the reader, we have mentioned this consideration in the 
caption to Table 1, as well as in the beginning of section IV. 

\end{enumerate}

\end{document}
