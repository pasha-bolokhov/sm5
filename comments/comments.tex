%\documentclass[12pt]{article}
\documentclass[12pt,preprintnumbers,nofootinbib]{revtex4}

\begin{document}

\begin{center}
\centering
\Large
Comments to the Paper
\end{center}


\begin{enumerate}
\item Page 2, ``The intriguing possibility ...'':
 This sentence needs to cite something,
 such as Kosteleck\'y and Samuel, PRD 39 683 (1989).
\item Page 2, ``Dimension three Lorentz...'':
 The terms shown use incorrect notation.
 The left $\psi$ in each term should be $\bar{\psi}$,
 and the factors of $i$ shouldn't be there.
\item Page 2, ``This is, of course, ...'':
 There are two problems with this sentence:
 \begin{enumerate}
 \item A citation is needed for the experimental bound quoted.
 \item The actual bounds that have been achieved directly involve nucleons, not
quarks.
 These probably induce bounds on quark parameters,
 but the exact nature of the quark bounds is an open question.
 \end{enumerate}
\item Page 7, ``Similarly, $f_{1,2}^{\mu\nu}$ cannot admix...'':
 This is a very awkward sentence and should be rewritten.
\item Page 14, ``Such interactions are dangerous...'':
 The second half of this sentence is obviously false --
 the authors discuss the dangerous interactions for the rest of the page,
 including three more numbered equations.
 They should not claim to exclude anything until they actually do so.
\item Page 14, equation (17):
 This equation has several problems.
 \begin{enumerate}
 \item Current experiments place different limits on the $b^\mu$ parameter
 for protons, neutrons, and electrons.
 Equation (17) may apply to at most one of these.
 \item The bound certainly isn't taken from citation [18] (Gagnon and Moore),
 which only deals with dimension-four-and-higher operators.
 It seems to come from a low-energy experiment.
 {\itshape Which} experiment is comes from depends on which particle species
 it's intended to refer to.
 \end{enumerate}
\item Page 14, ``Even in the very conservative assumption...'':
 It is not clear why the authors call potential efforts ``futile''.
 Is is because any effects associated with $\tilde{c}^\mu$ would be overwhelmed
 by other Lorentz-violating effects?
 Is is because the bound is far from rough estimates of the expected size of
$\tilde{c}^\mu$?
 The first reason is valid, but if true needs to be explained by the authors.
 The second reason is not valid because any estimated sizes are {\itshape very}
rough and unreliable.
 If either reason or something else pertains, it needs to be explained.
\item Page 16, Table 1:
 As with equation (17), the size of constraints depends on particle species.
 Moreover, many constraints deal directly with composite particles
 rather than quarks,
 and the full connection between the two is currently unknown.
\item Page 17, ``Known limits [20] on interaction...'':
 While the cited reference is relevant,
 stronger bounds on nuclear Lorentz-violating interactions
 arise from several more-recent experiments.
 I suggest that the authors investigate those.
\item Page 17, ``We comment that ...'':
 This point is true, and is important enough that is should be mentioned
earlier.
 Among other places, something like it should appear in the caption to Table 1
 so that readers who only skim the current paper do not get the wrong
impression.  
\end{enumerate}

\end{document}
